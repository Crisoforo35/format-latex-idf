\documentclass[12pt]{article}

\usepackage{anyfontsize}
\usepackage{graphicx}
\usepackage{geometry}
\usepackage{hyperref}
\usepackage{float}
\usepackage{pdfpages}
\usepackage{indentfirst}

\author{Ernesto Crisoforo Ortiz Burelo}
\title{Problemario 2 - Tema 2}
\date{05/10/2023}

\geometry{
  top=.5in,
  bottom=.5in,
  left=1.25in,
  right=1.25in
}
\begin{document}

\hypersetup{
  linkcolor=blue,
  urlcolor=blue,
  linktoc=all
}

\begin{figure}
  \includegraphics[width=0.2\textwidth]{logo_itsc.png}
\end{figure}
\centering
\fontsize{17}{50}\selectfont
\textbf{Instituto Tecnologico Superior de Comalcalco.}

\raggedright

\fontsize{16}{80}\selectfont
Problemario No. 1 -  Teorema fundamental del calculo.

\fontsize{16}{40}\selectfont
Fecha de la Realización de la práctica: Semana 4.

\fontsize{16}{40}\selectfont
\textbf{Nombre del estudiante}: Ernesto Crisoforo Ortiz Burelo.

\fontsize{16}{40}\selectfont
\textbf{No. de Control}: TE230602.

\fontsize{16}{40}\selectfont
\textbf{Carrera}: Ingeniería en Sistemas Computacionales.

\fontsize{16}{40}\selectfont
\textbf{Materia}: Calculo Integral.

\fontsize{16}{40}\selectfont
\textbf{Grado y grupo}: 2B.

\fontsize{16}{40}\selectfont
\textbf{Maestro}: M.C. Braly Guadalupe Peralta Reyes.

\raggedleft

\fontsize{15}{220}\selectfont
Comalcálco, Tabasco a 15/02/2024

\newpage

\raggedright

\begin{figure}
  \includegraphics[width=0.2\textwidth]{logo_itsc.png}
\end{figure}

\centering
\fontsize{17}{50}\selectfont
\textbf{Instituto Tecnologico Superior de Comalcalco.}

\raggedright

\fontsize{16}{60}\selectfont
Problemario No. 1 -  Teorema fundamental del calculo.

\fontsize{16}{35}\selectfont
Fecha de la Realización de la práctica: Semana 4.

\fontsize{16}{35}\selectfont
\textbf{Equipo No.} 4.

\fontsize{16}{35}\selectfont
\textbf{Equipo: \\}

\fontsize{16}{25}\selectfont
Omar Escalante Mendez - TE230596 \\
Mauricio Hernandez Olive - TE230596 \\
Jimmy del Carmen Hernández Méndez TE230601 \\
Juan Valentin Perez Diaz - TE230594 \\
Ernesto Crisoforo Ortiz Burelo - TE230602

\fontsize{16}{35}\selectfont
\textbf{Carrera}: Ingeniería en Sistemas Computacionales.

\fontsize{16}{35}\selectfont
\textbf{Materia}: Calculo Integral.

\fontsize{16}{35}\selectfont
\textbf{Grado y grupo}: 2B.

\fontsize{16}{35}\selectfont
\textbf{Maestro}: M.C. Braly Guadalupe Peralta Reyes.

\raggedleft

\fontsize{15}{100}\selectfont
Comalcálco, Tabasco a 15/02/2024

\newpage
\raggedright

\fontsize{17}{100}\selectfont
\textbf{Indice}

\raggedright

\fontsize{16}{105}\selectfont
\hyperref[rehearsal-title-y]{Ensayo Desarrollo Historico del Calculo Integral......................4}

\begin{quote}
  \raggedleft

  \fontsize{16}{30}\selectfont
  \hyperref[rehearsal-primary]{Principales Aportaciones....................................................5}

  \fontsize{16}{30}\selectfont
  \hyperref[rehearsal-end]{Conclusión..........................................................................5}
\end{quote}

\fontsize{16}{105}\selectfont
\hyperref[sec:sumatoria]{Sumatoria...................................................................................6}
\begin{quote}
  \raggedleft

  \fontsize{16}{30}\selectfont
  \hyperref[sumatoria:problema_1]{Problema 1.........................................................................6}

  \fontsize{16}{30}\selectfont
  \hyperref[sumatoria:problema_2]{Problema 2.........................................................................6}

  \fontsize{16}{30}\selectfont
  \hyperref[sumatoria:problema_3]{Problema 3.........................................................................7}

  \fontsize{16}{30}\selectfont
  \hyperref[sumatoria:problema_4]{Problema 4.........................................................................7}

  \fontsize{16}{30}\selectfont
  \hyperref[sumatoria:problema_5]{Problema 5.........................................................................7}
\end{quote}

\fontsize{16}{50}\selectfont
\hyperref[sec:fig_amorfas]{Figuras Amorfas.........................................................................8}

\begin{quote}
  \raggedleft

  \fontsize{16}{30}\selectfont
  \hyperref[fig_amorfas:problema_1]{Problema 6.........................................................................8}

  \fontsize{16}{30}\selectfont
  \hyperref[fig_amorfas:problema_2]{Problema 7.........................................................................8}

  \fontsize{16}{30}\selectfont
  \hyperref[fig_amorfas:problema_3]{Problema 8.........................................................................9}
\end{quote}


\newpage


\centering
\section*{\fontsize{14}{30}\selectfont DESARROLLO HISTORICO DEL CALCULO INTEGRAL} \label{rehearsal-title-y}



\fontsize{14}{15}\selectfont
\raggedright 
El cálculo integral es una rama de las matemáticas que se ocupa del estudio de las áreas, volúmenes, longitudes de curvas y otros conceptos relacionados. El cálculo integral es una herramienta fundamental en muchas áreas de las matemáticas, la ciencia y la ingeniería.  \newline

Los orígenes del cálculo integral se remontan a la antigüedad. En el siglo III a.C., el matemático griego Arquímedes desarrolló métodos para calcular el área de figuras planas y el volumen de sólidos. Estos métodos se basaban en la idea de dividir las figuras en partes más pequeñas y luego sumar las áreas o volúmenes de las partes. \newline

En el siglo XVII, los matemáticos Isaac Newton y Gottfried Leibniz desarrollaron de forma independiente el cálculo integral moderno. Newton utilizó el cálculo integral para resolver problemas de física, como el movimiento de los cuerpos celestes. Leibniz, por su parte, desarrolló una notación para el cálculo integral que se utiliza hasta la actualidad. \newline

En los siglos XVIII y XIX, el cálculo integral se desarrolló aún más. Los matemáticos Laplace, Fourier y Gauss desarrollaron métodos para calcular integrales definidas e indefinidas. También se desarrollaron métodos para calcular integrales de funciones complicadas. \newline

En el siglo XX, el cálculo integral se ha aplicado a nuevas áreas de la ciencia y la ingeniería. Por ejemplo, se ha utilizado para desarrollar modelos matemáticos de fenómenos físicos, como la propagación del calor y la difusión de la luz. También se ha utilizado para diseñar nuevos dispositivos, como los circuitos integrados. \newline


\newpage

\section*{Principales aportaciones al desarrollo del cálculo integral} \label{rehearsal-primary}

A continuación se presentan algunas de las principales aportaciones al desarrollo del cálculo integral: \newline

\begin{itemize}
  \item Arquímedes (siglo III a.C.): desarrolló métodos para calcular el área de figuras planas y el volumen de sólidos.
  \item Isaac Newton (siglo XVII): utilizó el cálculo integral para resolver problemas de física, como el movimiento de los cuerpos celestes.
  \item Gottfried Leibniz (siglo XVII): desarrolló una notación para el cálculo integral que se utiliza hasta la actualidad.
  \item Pierre-Simon Laplace (siglo XVIII): desarrolló métodos para calcular integrales definidas e indefinidas.
  \item Joseph Fourier (siglo XIX): desarrolló métodos para calcular integrales de funciones periódicas.
  \item Carl Friedrich Gauss (siglo XIX): desarrolló métodos para calcular integrales de funciones complicadas.
\end{itemize}
\section*{Conclusión} \label{rehearsal-end}

El cálculo integral es una herramienta fundamental en muchas áreas de las matemáticas, la ciencia y la ingeniería. El desarrollo histórico del cálculo integral ha sido un proceso gradual, en el que han contribuido muchos matemáticos a lo largo de los siglos.

\newpage

\fontsize{17}{100}\selectfont
\textbf{Problemario:}

\section*{Sumatoria}\label{sec:sumatoria}

\fontsize{15}{40}\selectfont
\section*{\normalfont 1) $\sum_{k = 1}^{7}$ \fontsize{20}{40}\selectfont $\frac{1}{k\ +\ 1}$ =} \label{sumatoria:problema_1}

\begin{quote}
  \fontsize{15}{25}\selectfont
  Primero sustituimos el número en la constante\\
  $\sum _{k=1}^7\:= \frac{1}{1+1} + \frac{1}{2+1} + \frac{1}{3+1} + \frac{1}{4+1} + \frac{1}{5+1} + \frac{1}{6+1} + \frac{1}{7+1} = $\\
  Hacemos las sumas correspondientes:\\
  $\sum _{k=1}^7\:= \frac{1}{2} + \frac{1}{3} + \frac{1}{4} + \frac{1}{5} + \frac{1}{6} + \frac{1}{7} + \frac{1}{8} = $\\
  Sumamos y reducimos términos:\\
  $\sum _{k=1}^7\:= \frac{1}{2} + \frac{1}{3} + \frac{1}{4} + \frac{1}{5} + \frac{1}{6} + \frac{1}{7} + \frac{1}{8} = 1 \frac{201}{280} $\\
  Este es nuestro resultado:\\
  $ 1 \frac{201}{280}$
\end{quote}

\fontsize{15}{40}\selectfont
\section*{\normalfont 2) $\sum_{m = 1}^{8}$ $(-1)^m$ $2^{m-2}$ =} \label{sumatoria:problema_2}

\begin{quote}
  \fontsize{15}{25}\selectfont
  Empezamos sustituyendo la literal: \\
  $\sum _{m=1}^8\:\left(-1\right)^m 2^{m-2}=((-1)^1 2^{1-2})+((-1)^2 2^{2-2})+((-1)^3 2^{3-2})+((-1)^4 2^{4-2})+((-1)^5 2^{5-2})+((-1)^6 2^{6-2})+((-1)^7 2^{7-2})+((-1)^8 2^{8-2})$
  Empezamos a reducir potencias:
  $\sum _{m=1}^8\:\left(-1\right)^2 2^{m-2}=(-1\cdot2^{-1})+(1\cdot 2^{0})+((-1\cdot 2^{1})+(1\cdot 2^{2})+(-1\cdot 2^{3})+(1\cdot 2^{4})+(-1\cdot 2^{5})+(1\cdot 2^{6})=$ \\
  Elevamos potencias:

  $(-1\cdot0.5)+(1\cdot1)+(-1\cdot2)+(1\cdot4)+(-1\cdot8)+(1\cdot16)+(-1\cdot32)+(1\cdot64)=$ \\
  Multiplicamos y sumamos los terminos:
  $-0.5+1-2+4-8+16-32+64=42.5$ \\
  El resultado es:
  $\sum _{m=1}^8\:\left(-1\right)^2 2^{m-2}=42.5$
\end{quote}


\newpage

\fontsize{15}{40}\selectfont
\section*{\normalfont 3) $\sum_{n = 1}^{6}$ $n$ $\cos(n\pi)$ =} \label{sumatoria:problema_3}


\begin{quote}
  \fontsize{12}{30}\selectfont
  = $\sum_{n = 1}^{6}$ = $  1 \cos(1\pi)  + 2 \cos(2\pi) + 3 \cos(3\pi) + 4 \cos(4\pi) + 5 \cos(5\pi) + 6 \cos(6\pi)$ \\
  = (-1) + 2 + (-3) + 4 + (-5) + 6 = 3
\end{quote}

\fontsize{15}{40}\selectfont
\section*{\normalfont 4) $\sum_{k = 3}^{7}$ \fontsize{17}{40} $\frac{(-1)^k\ 2^k}{(k\ +\ 1)}$ =} \label{sumatoria:problema_4}


\begin{quote}
  \fontsize{14}{30}\selectfont
  $\sum_{k = 3}^{7} $ = $ \frac{(-1)^3\ 2^3}{(3\ +\ 1)} + \frac{(-1)^4\ 2^4}{(4\ +\ 1)} + \frac{(-1)^5\ 2^5}{(5\ +\ 1)} + \frac{(-1)^6\ 2^6}{(6\ +\ 1)} + \frac{(-1)^7\ 2^7}{(7\ +\ 1)} $ \\
  $\sum_{k = 3}^{7} $ = $\frac{-8}{4} + \frac{16}{5} + \frac{-16}{3} + \frac{64}{7} + \frac{-128}{8} $ \\
  = \fontsize{12}{30}\selectfont $(-2) + \frac{16}{5} + \frac{-16}{3} + \frac{64}{7} + (-16)$ = $\frac{-1154}{105}$
\end{quote}

\fontsize{15}{40}\selectfont
\section*{\normalfont 5) $\sum_{k = -1}^{6}$ $k$ $\sin(k\pi/2)$ =} \label{sumatoria:problema_5}


\begin{quote}
  \fontsize{12}{20}\selectfont
  $\sum_{k = -1}^{6} $ = $ 1 \sin(-1\pi/2) + 0 \sin(0\pi/2) + 1 \sin(1\pi/1) + 2 \sin(2\pi/2) +$
  $3 \sin(3\pi/2) + 4 \sin(4\pi/2) + 5 \sin(5\pi/2) + 6 \sin(6\pi/2)$
  = 4
\end{quote}


\newpage

\section*{\fontsize{17}{20}\selectfont Figuras Amorfas}\label{sec:fig_amorfas}

\fontsize{15}{40}\selectfont
\section*{\normalfont 6) y = $x$ + 2; $a$ = 0, $b$ = 1} \label{fig_amorfas:problema_1}

\begin{quote}
  \fontsize{14}{30}\selectfont
  $\int_{0}^{1} x + 2 \,dx$ = $\int x\ dx$ + $\int 2\ dx$ \\
  $\int x\ dx $ = $\frac{x^2}{2} + c$ \\
  $\int 2\ dx $ = $2x + c$ \\
  $\int_{a}^{b} f(x)\ dx = F(b) - F(a)$ \\
  $ \frac{1^2}{2} + 2(1)  -  \frac{0^2}{2} + 2(0)$ \\
  $\frac{1}{2} + 2 - 0$ \\
  $\frac{5}{2}$ = 2.5
\end{quote}

\fontsize{15}{40}\selectfont
\section*{\normalfont 7) y = 2$x$ + 2; $a$ = -1, $b$ = 1. Sugerencia: $x_i$ = -1 + \fontsize{17}{40}\selectfont $\frac{2i}{n}$} \label{fig_amorfas:problema_2}


\begin{quote}
  \fontsize{14}{30}\selectfont
  $\int_{-1}^{1} 2x + 2 \,dx$ = $\int 2x\ dx$ + $\int 2\ dx$ \\
  $\int x\ dx $ = $x^2 + c$ \\
  $\int 2\ dx $ = $2x + c$ \\
  $\int_{a}^{b} f(x)\ dx = F(b) - F(a)$ \\
  $ 1^2 + 2(1)  -  (-1)^2 + 2(-1)$ \\
  $(1 + 2) - (1 -2) $ \\
  = 4
\end{quote}


\newpage

\fontsize{15}{40}\selectfont
\section*{\normalfont 8) y = $x^3$ + x; $a$ = 0, $b$ = 1} \label{fig_amorfas:problema_3}

\begin{quote}
  $\int_{0}^{1} x^3 + x \,dx$ = $\int x^3\ dx$ + $\int x\ dx$ \\
  $\int x^3\ dx $ = $\frac{x^4}{4} + c$ \\
  $\int x\ dx $ = $\frac{x^2}{2} + c$ \\
  $\int_{a}^{b} f(x)\ dx = F(b) - F(a)$ \\
  $(\frac{1^4}{4} + \frac{1^2}{2})  -  (\frac{0^4}{4} + \frac{0^2}{2})$ \\
  $\frac{3}{4} - 0 $ \\
  = 0.75
\end{quote}

\newpage

\includegraphics[width=\textwidth]{./img/Examen_1_integral_semana_4.jpeg}

\newpage

\includegraphics[width=\textwidth]{./img/Examen_2_integral_semana_4.jpeg}

\newpage

\includegraphics[width=\textwidth]{./img/Examen_3_integral_semana_4.jpeg}

\end{document}